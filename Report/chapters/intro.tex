\documentclass[../main-report.tex]{subfiles}

\begin{document}

\part*{MỞ ĐẦU}
\addcontentsline{toc}{chapter}{MỞ ĐẦU}
\section*{Lý do chọn đề tài}
Python là một ngôn ngữ lập trình bậc cao cho các mục đích lập trình đa năng, do Guido
van Gusso tạo ra và lần đầu ra mắt vào năm 1991. Python được thiết kế với ưu điểm mạnh là dễ
đọc, dễ học và dễ nhớ. Python là ngôn ngữ có hình thức rất sáng sủa, cấu trúc rõ ràng, thuận tiện
cho người mới học lập trình và là ngôn ngữ lập trình dễ học.

Hiện nay, sự bùng nổ của kỷ nguyên số về các lĩnh vực như Trí Tuệ Nhân Tạo (Artificial
Intelligence) và Dữ Liệu Lớn (Big Data) đã góp phần gia tăng không nhỏ nhu cầu sử dụng
Python trong những năm gần đây và còn tiến xa hơn nữa trong tương lai.

Nội dung gồm 3 phần chính:

\begin{itemize}
\item \textbf{Tìm hiểu về ngôn ngữ lập trình python} - tìm hiểu một cách tổng quan về ngôn ngữ lập trình python và một số thư viện của nó.
\item \textbf{Thuật toán tìm đường đi ngắn nhất} - phần này tìm hiểu và cài đặt 2 thuật thuật toán tìm đường đi ngắn nhất trên đồ thị vô hướng là thuật toán Dijkstra và Bellman-Ford.
\item \textbf{Thuật toán cây quyết định} - thực hiện tìm hiểu về một thuật toán trong Machine Learning là thuật toán cây quyết định.
\end{itemize}

\section*{Mục đích thực hiện đề tài}
Khi thực hiện đề tài, em mong muốn được tiếp cận và tìm hiểu ngôn ngữ lập trình Python và một số thư viện của python cũng như hiểu và cài đặt một số thuật toán cơ bản trên Python. Và từ đó vận dụng vào ngành đang học - Hệ thống thông tin và truyền thông.

Hai mục tiêu hướng đến là:

\begin{itemize}
\item Thứ nhất, sẽ có kiến thức cơ bản về Python và các thư viện trên Python.
\item Thứ hai, tìm hiểu và có thể cài đặt một số thuật toán cơ bản đã được học hoặc thuật toán mới.
\end{itemize}

\end{document}
